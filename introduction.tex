\section{introduction}

Stereo is an intensively and extensively studied problem in computer vision. One popular approach for getting the 3D model is by first calculating the depthmap, and then doing the fusion. The fusion step is able to remove the noise by exploring the redundant information in the depthmaps. This paper mainly focuses on depthmap calculation. The local depthmap estimation method is often performed using patch based methods. The method typically search in the whole candidate space and find the best match. It has the advantage that high precision can be reached in textured area and can be easily parallelized. Global method can make textureless areas smoother. Depthmap calculation also involves two images or multiple images. Comparing to two-view stereo, using multiple images to calculate depthmap helps reduce ambiguity and increase precision. 

PatchMatch for stereo is popular recently. PatchMatch algorithm was initially introduced as a computationally efficient way to compute a nearest neighbour field. between two images. The PatchMatch stereo uses the information that nearby pixels on the reference image should have similar depth, so that not all depth candidates are tested. xxx further accelerates the method by changing the propagation scheme so that each rows/cols can be parallelized in each sweep. This paper extends the binocular PatchMatch method into multiple view stereo.

Many multiview algorithms for depthmap calculations require the neighboring images of the reference view are available. Normally those images are extracted from a video stream or manually picked up. If the neighboring images are available, to calculate the depth for a given pixel in the reference image, not all target images are useful due to occlusion, different view perspective, non-Lambertian reflection, etc. However, in case of large unstructured image set, finding the neighboring images is a problem. For example, the internet collected image set is very complex. It may contain huge number of images taken from different positions, with different camera exposure, different illuminations, various foreground object occlusion, etc. How to explore useful information efficiently and effectively within the large image set to calculate depthmap becomes more difficult.

The main contribution of this paper is that we provide a scheme to cleverly select target images for each pixel at the same time of doing stereo. The goal is achieved based on the observation that neighboring pixels on the reference image are like to have the same set of correct target images. In our method, we keep a selection probability map for each target image as shown in Fig. This probability map guides on the pixel level if the target image should be selected. For example, to calculate the depth of a pixel $p$ in the reference image, the value $S_p^i$ at the position of $p$ on probability map $i$ shows the probability if target image $i$ should be selected. The selection probability map is initialized randomly and updated online. The images that are used for depth testing are sampled based on the selection probablity maps.

The rest of the paper is laid out as follows: Some previous works are introduce in section 2. We describe our method in detail in section 3. Section 4 shows the experimental results.

